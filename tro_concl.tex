
\paragraph{Benefits of Policy Composition}
Since the hybrid control policy is based on local feedback control policies, the
overall controller inherits the properties of the individual policies.  This allows
the individual policies to be tuned to local conditions, whether to provide safety or
enhanced performance.  The local policies, in order to be composable, have provable
convergence guarantees, and retain the robustness to disturbances that is the
hallmark of feedback control.  The local feedback control policies are designed to be
memoryless, and allow for real time control.  Because each local policy has an
explicit domain, the hybrid control approach is inherently safe.  If a disturbance
takes the system outside the domains of all policies, the robot is halted and
execution of the hybrid control policy terminates.

By planning in the space of control policies using the prepares graph, the planning
becomes very flexible with regard to task.  This approach opens the door to formal
synthesis of hybrid feedback controllers for complex systems; for example the parking
controller demonstrated in \rechap{planning}.
%The size of the prepares graph is typically smaller than that required by a
%conventional grid-based planner for the same navigation problem; thus, the speed of
%planning and re-planning will be faster.  
The approach allows analysis of the reachability of a goal, or realizability of a
specification, with a given policy suite during the discrete planning phase, prior to
execution.

Using automata to execute the local feedback policies allow the
systems to exhibit complex, multi-task behaviors.  The approach
enables tasks to be specified at a high-level, and then executed in a
continuous manner, using a hybrid control policy synthesized from the
suite of local control policies and associated prepares graph.
Repetitive tasks are naturally encoded in the framework, which allows
the approach to induce limit cycle type behaviors.


\subsection{Extension of Planning Tools}
\label{sec:extend_planning}

The demonstrations provided in this thesis highlight the power and flexibility of the
policy composition approach.  They also point to several shortcomings that should be
addressed in the discrete planning domain.  Addressing these issues will increase the
power of the policy composition approach for the policies and systems described
above.  
%In some cases, there may be existing techniques that may be brought to bear
%on synthesis of hybrid control policies for robotics.

\paragraph{Sensor Automata and Composition}
The automata synthesis approach described in \rechap{planning} depends on sensors
that provide binary signals to the automata during run time.  As with the `Hazard'
sensor, these ``sensors'' are often hybrid automata in their own right.  That is, the
binary sensor values depend on a mixture of continuous variables and discrete logic.
Defining these sensors is currently done on an \emph{ad hoc} basis.  Techniques are
needed to synthesize these sensor automata, and prove that the composition of
multiple sensor and control automata are valid, and preserve the desired
specifications and liveness conditions.  An example, taken from the simulations in
this thesis and the DARPA Urban Grand Challenge (UGC), is the need to resolve
precedence at intersections and four-way stops.  Formal synthesis methods coupled
with local policy composition offers a way to automate what is currently a labor
intensive, error prone process; as evidenced by the failures of most of the teams
that entered the DARPA UGC.

\paragraph{Formal Recovery Methods and Global Synthesis}
Disturbances are a fact of life.  Thus, the automata synthesis should include all
available policies to maximize the domain over which the automata is valid, and have
a formal method of recovery.  For the repetitive behaviors, such as patrolling,
demonstrated in this thesis, it is sufficient to augment the automata with unused
policies, and then search for a node that had the correct discrete sensor values and
whose associated policy contained the current pose.  A more complex scenario, such as
the mail delivery robot, will require a more formal recovery approach to allow the
system to recover gracefully.  Instead of an \emph{ad hoc} approach, a formal and
automated approach to defining a recovery strategy is desired.  The recovery approach
should also include all policies to maximize the domain.

\paragraph{Automata Synthesis with Heuristic Costs}
A major shortcoming of the automata synthesis approaches demonstrated in this thesis
is that they do not consider heuristic costs.  Within a given policy suite, there may
be many policy combinations that will address a given scenario.  In fact, this is
desirable for maximum planning flexibility.  The synthesis approach needs to be able
rank different policy choices based on their relative cost.  Current techniques only
consider the number of transitions made, that is the number of edges traversed in the
graph walk, when choosing policies.  The focus of current techniques is on dealing
with the state explosion problem using efficient data structures such as Binary
Decision Diagrams (BDD)~\cite{Bryant_86,clarke_99}.  There has been some work in
combining BDDs with heuristic search; for example, consider the Set\{A*\}
approach~\cite{jensen_02}.  That work may prove fruitful for automata synthesis
research.

\paragraph{Automata Synthesis with Non-deterministic Outcomes}
The automata synthesis used in this thesis only considered deterministic prepares
graphs.  This eliminated the use of the extended prepares relationship.  There has
been some work on defining sequence based approaches which allow non-deterministic
prepares graphs~\cite{kloetzer_07}; however, the synthesis techniques for reactive
automata do not allow non-deterministic transitions at this time.

\paragraph{Automata Synthesis with Hybrid Stability Analysis}
The stability of the order-based hybrid policies is based on the assumption of
monotonic policy switching.  With automata, limit cycles are allowed.  In the hybrid
systems community, it is well known that switching between stable vector fields can
induce instability~\cite{Branicky:98b,decarlo_00,liberzon_99}.  While this is not an
issue for the kinematic systems addressed in this thesis, stability analysis will be
fundamental to more complex systems.  There are several approaches to analyzing the
stability of existing hybrid systems~\cite{Branicky:98b,decarlo_00,liberzon_99};
however, going in reverse, the synthesis problem must address this issue during
construction.

\vspace{0.5in}

To conclude, this thesis advocates an approach to specifying robot controllers that
respects low-level constraints by design, and provides a natural interface for
specifying user intentions.  Low-level feedback control policies induce local
behaviors in a guaranteed manner.  The user interacts at a high-level to specify
intended behaviors.  The robot then automatically synthesizes a hybrid control policy
that can realize the intention, or reports that the goal is not realizable for the
current collection of policies and initial condition.  The hybrid control policy
inherits the desirable properties of the local feedback policies, while guaranteeing
the high-level behaviors.  This approach, which is demonstrated on a class of
nonholonomically constrained systems in this thesis, is widely applicable, and
likely necessary for the dynamically capable robots of the future.



